\section{Requêtes}

Nous avons interrogé notre entrepôt de données via quelques requêtes, détaillées ici.
Note: Pour des raisons de lisibilité, les résultats comportant trop de lignes ont été tronqués.

\subsection{Affichage des films pour adultes}
\subsubsection{Requête}
\begin{lstlisting}
SELECT fi.title
FROM fait fa, d_film fi, d_genre ge
WHERE fa.d_genre_id=ge.id
		AND fa.d_film_id = fi.id
		AND ge.adult = 1;
\end{lstlisting}
Commençons avec une requête simple pour prendre en main notre base de données.
\subsubsection{Résultat}
\begin{lstlisting}
TITLE
-------------------------------------------------
Rocco - Perfect Girls 5
\end{lstlisting}

\subsection{Entrées en fonction des compagnies et du genre}
\subsubsection{Requête}
\begin{lstlisting}
SELECT ge.genre_name, co.name_, sum(fa.admissions) AS "nb_entrees"
FROM fait fa, d_genre ge, d_company co
WHERE fa.d_genre_id = ge.id
		AND fa.d_company_id = co.id
		AND ge.genre_name !='NULL'
		AND co.name_ !='NULL'
GROUP BY ROLLUP(ge.genre_name,co.name_);
\end{lstlisting}
Donne le total des entrées pour chaque genre pour chaque compagnie de production, mais aussi pour chaque genre en tout
\subsubsection{Résultat}
\begin{table}[]
	\centering
	\texttt{
	\begin{tabular}{l l l}
		GENRE\_NAME & NAME_     & nb\_entrees \\
		\hline
Crime & Televisi??n Espa??ola TVE	& 6701026 \\
Crime &								& 6701026 \\
Drama & El Deseo					& 1617375 \\
Drama & DreamWorks					& 14143624 \\
Drama & V??a Digital				& 8526951 \\
Drama & Hotshot Films				& 0 \\
Drama & United Artists				& 14871212 \\
Drama & France 3 Cin??ma			& 0 \\
Drama & P.P. Film Polski			& 0 \\
Drama & Fine Line Features			& 6665199 \\
Drama & Detour Film Production		& 921348 \\
	\end{tabular}}
\end{table}


\subsection{TOP 10 des films qui ont généré le plus de revenu}
\subsubsection{Requête}
\begin{lstlisting}
SELECT fi.title
FROM d_film fi, fait fa
WHERE fa.d_film_id = fi.id
		AND ROWNUM <= 10
ORDER BY fa.revenue;
\end{lstlisting}

\subsubsection{Résultat}
\begin{lstlisting}
TITLE
-------------------------------------------------
Three Colors: Blue
Italian for Beginners
The Rolling Stones: Gimme Shelter
No End
Run Lola Run
Lock, Stock and Two Smoking Barrels
Match Point
Princess Mononoke
The Lord of the Rings: The Fellowship of the Ring
Finding Nemo

\end{lstlisting}

\subsection{Entrées d'argent par genres en fonction de l’année}
\subsubsection{Requête}
\begin{lstlisting}
SELECT dt.year, dg.genre_name, SUM(f.revenue)
FROM fait f, d_time dt, d_genre dg
WHERE f.d_time_id = dt.id
		AND f.d_genre_id = dg.id
GROUP BY CUBE (dt.year, dg.genre_name);
\end{lstlisting}
Cette requête nous permet d'avoir un regard assez complet sur les revenus des films, en croisant les genres et les années. Grâce à l'operateur \texttt{CUBE}, nous avons aussi les revenus sur les années tout genres confondus et sur les genres, indépendamment des années.
\subsubsection{Résultat}
\begin{table}[]
	\centering
	\texttt{
	\begin{tabular}{l l l}
		YEAR & GENRE\_NAME     & SUM(F.REVENUE)\\
		\hline
		& 					& 0 \\
		& 					& 3260793431 \\
		& NULL 				& 0 \\
		& Crime 			& 40266982 \\
		& Drama 			& 514209432 \\
		& Action 			& 488181621 \\
		& Comedy 			& 37856188 \\
		& Adventure 		& 1030743672 \\
		& Animation 		& 940335536 \\
		& Documentary 		& 0 \\
		& Science Fiction 	& 209200000 \\
2000	& 					& 40031879 \\
2000	& Drama 			& 40031879 \\
2000	& Comedy 			& 0 \\
	\end{tabular}}
\end{table}

\subsection{Réussite moyenne (entrées, revenu, popularité, vote moyens) et nombre de films des boites de production, par trimestre}
\subsubsection{Requête}
\begin{lstlisting}
SELECT dt.season, dt.year, dc.name_, AVG(f.admissions),
	AVG(f.popularity), AVG(f.revenue), AVG(f.vote_average),
	count(f.id)
FROM fait f, d_time dt, d_company dc
WHERE f.d_time_id = dt.id
		AND f.d_company_id = dc.id
GROUP BY ROLLUP(dc.name_, dt.year, dt.season);
\end{lstlisting}
Ici, nous demandons un compte rendu complet de la réussite moyenne des compagnies de production. Chaque indicateur (moyenne des entrées, revenu, popularité, votes) est calculé non seulement pour chaque entreprise, mais aussi pour chaque entreprise pour chaque année (où au moins un film à été produit), et pour chaque année pour chaque saison, grâce au fonctionnement de l'opérateur \texttt{ROLLUP}
\subsubsection{Résultat}
\begin{table}[]
	\centering
	\texttt{
	\begin{tabular}{l l l}
		SE& YEAR& NAME_ & AVG(F.ADMISSIONS)& AVG(F.POPULARITY)& AVG(F.REVENUE)& AVG(F.VOTE_AVERAGE)& COUNT(F.ID)\\
		\hline
   &1987 & Paramount Pictures		& 49739707		& 8.0614	& 299965036	& 6.1 	& 1 \\
SP &1989 & Paramount Pictures		& 11769827		& 9.919992	& 70200000	& 5.6 	& 1 \\
   &1989 & Paramount Pictures		& 11769827		& 9.919992	& 70200000	& 5.6 	& 1 \\
   &     & Paramount Pictures		& 21572331.3	& 9.198693	& 129666259	& 6.375 & 4 \\
WI &1998 & Handmade Films Ltd.		& 4702374		& 4.880012	& 28356188	& 7.5 	& 1 \\
   &1998 & Handmade Films Ltd.		& 4702374		& 4.880012	& 28356188	& 7.5 	& 1 \\
   &     & Handmade Films Ltd.		& 4702374		& 4.880012	& 28356188	& 7.5 	& 1 \\
SU &2002 & Les Films du Losange		& 0				& 1.735325	& 0			& 6.8 	& 1 \\
   &2002 & Les Films du Losange		& 0				& 1.735325	& 0			& 6.8 	& 1 \\
   &     & Les Films du Losange		& 0				& 1.735325	& 0			& 6.8 	& 1 \\
SU &1998 & X-Filme Creative Pool	&1211807		& 7.326948	& 7267585	& 7.2 	& 1 \\
	\end{tabular}}
\end{table}


\subsection{Classement des genres qui engendrent un temps moyen a l'affiche le plus long}
\subsubsection{Requête}
\begin{lstlisting}
SELECT ge.genre_name,avg(fa.runtime) AS "Moyenne de temps en salle (j)", rank() over (order by avg(fa.runtime) desc) AS "Rang"
FROM fait fa, d_genre ge
WHERE fa.d_genre_id = ge.id
GROUP BY ge.genre_name;
\end{lstlisting}

\subsubsection{Résultat}
\begin{table}[]
	\centering
	\texttt{
	\begin{tabular}{| l | l | l |}
		\hline
		GENRE\_NAME     & Durée Moyenne en salle (en jours) & Rang \\
		\hline
		NULL            & 210                           & 1  \\
		Adventure       & 156                           & 2 \\
		Science Fiction & 119.5                         & 3  \\
		Drama & 110.07 & 5 \\
		Comedy & 106.66 & 6 \\
		Crime & 106 & 6 \\
		Action & 97.33 & 7 \\
		Documentary & 66.5 & 8 \\
		Animation & 60.5 & 9 \\
		\hline
	\end{tabular}
}
\end{table}

\subsection{Cumul des budgets des films francais depuis 2000}
\subsubsection{Requête}
\begin{lstlisting}
SELECT ti.year, sum(fa.budget) AS, sum(sum(fa.budget)) over(order by ti.year rows unbounded preceding)
FROM fait fa, d_time ti, d_zone zo
WHERE fa.d_time_id = ti.id AND fa.d_zone_id = zo.id AND ti.year >= 2000 AND zo.original_language = 'fr'
GROUP BY ti.year;
\end{lstlisting}
\subsubsection{Résultat}
\begin{lstlisting}
no rows selected

\end{lstlisting}

\subsection{Categorisation des pays où sont produit les films generant le plus de revenu par an avec leur films}
\subsubsection{Requête}
\begin{lstlisting}
SELECT zo.production_country, ti.year, NTILE(4) over(order by sum(fa.revenue) desc) as "Groupe"
FROM fait fa, d_zone zo, d_time ti
WHERE fa.d_zone_id = zo.id AND fa.d_time_id = ti.id
GROUP BY ti.year,zo.production_country;

\end{lstlisting}
\subsubsection{Résultat}
\begin{lstlisting}
PRODUCTION_COUNTRY	       YEAR	Groupe
------------------------ ---------- ----------
United States of America       2003	     1
New Zealand		       2001	     1
United States of America       1987	     1
United States of America       1979	     1
Japan			       1997	     1
Ireland 		       2005	     1
United States of America       1989	     1
Spain			       2002	     1
Spain			       2004	     2
Argentina		       2000	     2
United Kingdom		       1998	     2

PRODUCTION_COUNTRY	       YEAR	Groupe
------------------------ ---------- ----------
United States of America       2004	     2
Canada			       2003	     2
United States of America       1961	     2
Germany 		       1998	     2
Austria 		       1995	     2
Germany 		       1987	     3
Austria 		       1998	     3
France			       1993	     3
Denmark 		       2000	     3
United States of America       1970	     3
United Kingdom		       1985	     3

PRODUCTION_COUNTRY	       YEAR	Groupe
------------------------ ---------- ----------
			       1921	     3
France			       1896	     3
Denmark 		       2002	     4
Poland			       1985	     4
United States of America       1932	     4
					     4
Japan			       1995	     4
Spain			       2005	     4
Austria 		       2006	     4
			       1938	     4

32 rows selected.

\end{lstlisting}

\begin{table}[H]
	\centering
	\texttt{
	\begin{tabular}{| l | l | l |}
		\hline
		PRODU & 1 & 1 \\
		\hline
	\end{tabular}
}
\end{table}

\chapter{Annexe}

\section{Script de récolte des données}
\subsection{Requêtes à l'API}
\begin{lstlisting}
#! /usr/local/bin/python3

import urllib.request
import urllib.parse
import json

movies_id = [] # Trop long pour le rapport
movies_array = []
api_key = "40d11b52ba315a0c925a73ee719f595d"

for id in movies_id:
	request = "https://api.themoviedb.org/3/movie/"+ id + "?api_key=" + api_key 
	print("issuing request: " ,request)
	
	results = urllib.request.urlopen(request)
	file_w = open("./movies/" + id + ".json","w+")
	file_w.write(results.read().decode("utf-8"))
	print(results.read().decode("utf-8"))
	movies_array.append(results.read().decode("utf-8"))
	file_w.close()
\end{lstlisting}

\subsection{Compilation des données dans un seul fichier}
Afin d'avoir des données du nombre d'entrées, nous en générons aléatoirement en se basant sur le chiffre d'affaire du film.

\begin{lstlisting}
#! /usr/local/bin/python3

import os, json, random

directory_in_str = "/Users/hippolytedubois/Documents/MASTER/M1/BDD/Projet_OLAP/API/movies"
directory = os.fsencode(directory_in_str)

movies_arr = []

for file in os.listdir(directory):
	filename = os.fsdecode(file)
	f = open(directory_in_str + '/' + filename)
	json_o = json.loads(f.read())
	json_o["admissions"] = int(json_o["revenue"] / 6 + ((json_o["revenue"] / 500) * (random.random() - 0.5)))
	movies_arr.append(json_o)
	f.close()

f = open("every_movies.json","w+")
f.write(json.dumps(movies_arr, sort_keys=True, indent=4, separators=(',', ': ')))
f.close()
\end{lstlisting}

\section{Création de la base de donnée}

\begin{lstlisting}
CREATE TABLE d_zone (
	id NUMBER PRIMARY KEY,
	original_language VARCHAR(2),
	production_country VARCHAR(24)
);

CREATE TABLE d_time (
	id NUMBER PRIMARY KEY,
	release_date DATE,
	month NUMBER, -- 01, 02...12
	season VARCHAR(2), -- FA, WI, SP, SU
	year NUMBER,
	decade NUMBER -- 1910, 1920... 2010
);

CREATE TABLE d_genre (
	id NUMBER PRIMARY KEY,
	genre_name VARCHAR(15),
	adult NUMBER(1,0) -- Adult = 1, kid = 0
);

CREATE TABLE d_film (
	id NUMBER PRIMARY KEY,
	title VARCHAR(49),
	overview VARCHAR(955),
	imdb_id VARCHAR(9),
	original_title VARCHAR(49),
	tagline VARCHAR(136),
	status_ VARCHAR(8)
);

CREATE TABLE d_company (
	id NUMBER PRIMARY KEY,
	name_ VARCHAR(43)
);

CREATE TABLE fait (
	id NUMBER PRIMARY KEY,
	admissions NUMBER,
	popularity FLOAT,
	revenue NUMBER,
	runtime NUMBER,
	budget NUMBER,
	vote_average FLOAT,
	vote_count NUMBER,
	
	-- FOREIGN KEYS
	d_zone_id NUMBER, -- origin
	d_time_id NUMBER, -- release date
	d_genre_id NUMBER,
	d_film_id NUMBER,
	d_company_id NUMBER,
	
	CONSTRAINT fk_d_zone FOREIGN KEY (d_zone_id) REFERENCES d_zone(id),
	CONSTRAINT fk_d_time FOREIGN KEY (d_time_id) REFERENCES d_time(id),
	CONSTRAINT fk_d_genre FOREIGN KEY (d_genre_id) REFERENCES d_genre(id),
	CONSTRAINT fk_d_film FOREIGN KEY (d_film_id) REFERENCES d_film(id),
	CONSTRAINT fk_d_company FOREIGN KEY (d_company_id) REFERENCES d_company(id)
);

\end{lstlisting}

\subsection{Conversion des données au modèle de la base de donnée }
Ce script Python génère un fichier d'instructions SQL qui remplit la base de donnée.
On remplie un fichier au moyen d'une redirection du stdout shell.
\begin{lstlisting}
#! /usr/local/bin/python3

import os, json

fname = "../API/every_movies.json"
seasons_arr = ["WI", "SP", "SU", "FA"]

insert_zone = "INSERT INTO d_zone(id, original_language, production_country) VALUES ({}, {}, {});\n"
insert_time = "INSERT INTO d_time(id, release_date, month, season, year, decade) VALUES ({}, to_date({} , \'YYYY-MM-DD\'), {}, {}, {}, {});\n"
insert_genre = "INSERT INTO d_genre(id, genre_name, adult) VALUES ({}, \'{}\', {});\n"
insert_film = "INSERT INTO d_film(id, title, overview, imdb_id, original_title, tagline, status_) VALUES ({}, \'{}\', \'{}\', \'{}\', \'{}\', \'{}\',  \'{}\');\n"
insert_company = "INSERT INTO d_company(id, name_) VALUES ({}, \'{}\');\n"
insert_fait = "INSERT INTO fait(id, admissions, popularity, revenue, runtime, budget, vote_average, vote_count, d_zone_id, d_time_id, d_genre_id, d_film_id, d_company_id) VALUES ({}, {}, {}, {}, {}, {}, {}, {}, {}, {}, {}, {}, {});\n"

if os.path.exists(fname):
with open(fname, 'r') as f:
json_arr = json.load(f)
id = 0
for movie in json_arr:
# Zone insert
country = "NULL"
if movie['production_countries']:
country = "\'" + movie['production_countries'][0]['name'] + "\'"
print(insert_zone.format(id, "\'" + movie['original_language'] + "\'", country))

# Date insert
date = "NULL"
month = "NULL"
season = "NULL"
year = "NULL"
decade = "NULL"
if movie['release_date']:
date = "\'" + movie['release_date'] + "\'"
month = movie['release_date'].split('-')[1]
season = "\'" + seasons_arr[int(movie['release_date'].split('-')[1]) // 4] + "\'"
year = movie['release_date'].split('-')[0]
decade = movie['release_date'].split('-')[0][:3] + "0"
print(insert_time.format(id, date, month, season, year, decade))

# Genre insert
genre_name = "NULL"
if movie['genres']:
genre_name = movie['genres'][0]['name']
adult = "0"
if movie['adult']:
adult = "1"
print(insert_genre.format(id, genre_name, adult))

# Film insert
print(insert_film.format(id, movie['title'], movie['overview'], movie['imdb_id'], movie['original_title'], movie['tagline'], movie['status']))

# Company insert
company = "NULL"
if movie['production_companies']:
company = movie['production_companies'][0]['name']
print(insert_company.format(id, company))

# Fact insert
print(insert_fait.format(id, movie['admissions'], movie['popularity'], movie['revenue'], movie['runtime'], movie['budget'], movie['vote_average'], movie['vote_count'], id, id, id, id, id))

id+=1
\end{lstlisting}

\subsection{Exemple de commande INSERT ainsi générées}

\begin{lstlisting}
INSERT INTO d_zone(id, original_language, production_country) VALUES (0, 'en', 'United Kingdom');

INSERT INTO d_time(id, release_date, month, season, year, decade) VALUES (0, to_date('1998-03-05' , 'YYYY-MM-DD'), 03, 'WI', 1998, 1990);

INSERT INTO d_genre(id, genre_name, adult) VALUES (0, 'Comedy', 0);

INSERT INTO d_film(id, title, overview, imdb_id, original_title, tagline, status_) VALUES (0, 'Lock, Stock and Two Smoking Barrels', 'A card sharp and his unwillingly-enlisted friends need to make a lot of cash quick after losing a sketchy poker match. To do this they decide to pull a heist on a small-time gang who happen to be operating out of the flat next door.', 'tt0120735', 'Lock, Stock and Two Smoking Barrels', 'A Disgrace to Criminals Everywhere.',  'Released');

INSERT INTO d_company(id, name_) VALUES (0, 'Handmade Films Ltd.');

INSERT INTO fait(id, admissions, popularity, revenue, runtime, budget, vote_average, vote_count, d_zone_id, d_time_id, d_genre_id, d_film_id, d_company_id) VALUES (0, 4702374, 4.880012, 28356188, 105, 1350000, 7.5, 1678, 0, 0, 0, 0, 0);

INSERT INTO d_zone(id, original_language, production_country) VALUES (1, 'de', 'Germany');

INSERT INTO d_time(id, release_date, month, season, year, decade) VALUES (1, to_date('1998-08-20' , 'YYYY-MM-DD'), 08, 'SU', 1998, 1990);

INSERT INTO d_genre(id, genre_name, adult) VALUES (1, 'Action', 0);

INSERT INTO d_film(id, title, overview, imdb_id, original_title, tagline, status_) VALUES (1, 'Run Lola Run', 'Lola receives a phone call from her boyfriend Manni. He lost 100,000 DM in a subway train that belongs to a very bad guy. She has 20 minutes to raise this amount and meet Manni. Otherwise, he will rob a store to get the money. Three different alternatives may happen depending on some minor event along Lolas run.', 'tt0130827', 'Lola rennt', 'Every second of every day youre faced with a decision that can change your life.',  'Released');

INSERT INTO d_company(id, name_) VALUES (1, 'X-Filme Creative Pool');

INSERT INTO fait(id, admissions, popularity, revenue, runtime, budget, vote_average, vote_count, d_zone_id, d_time_id, d_genre_id, d_film_id, d_company_id) VALUES (1, 1211807, 7.326948, 7267585, 81, 1530000, 7.2, 678, 1, 1, 1, 1, 1);

\end{lstlisting}
